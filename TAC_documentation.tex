\documentclass[letter paper, 12pt]{article}
\usepackage[english]{babel}
\usepackage{amsmath}
\usepackage{amsfonts}
\usepackage{amssymb}
\usepackage{gensymb}
\usepackage{changepage}
\usepackage[margin=1in]{geometry}
\usepackage{epsfig}
\usepackage{enumerate}
\usepackage{mathtools}
\usepackage{wasysym}
\usepackage{graphicx}
\usepackage[normalem]{ulem}
\usepackage{setspace}

\newcommand{\tab}{${}_{}$\hspace{0.2in}}
\newcommand{\noin}{\noindent}

\begin{document}
\noindent Thomas Finn Lidbetter and Patrick Coyle\\
Comp 3721\\
Dr. Ricker\\
TAC Milestone 4\\
\today \bigskip
\begin{center}
	TAC Documentation
\end{center}
\noin \underline{Reasons for selecting TAC:}\\
\tab We chose to use Finn and Michael's TickAttack project largely because this already made use of more design patterns than Patrick and Lucas' game. Furthermore TAC seemed to be better organised than the alternative and more extendable in most aspects. We did, however, find that adding a new quest was more difficult than ti might have been in Patrick and Lucas' implementation. More details on this will be given in the section on quest incorporation. In addition to this, while we had possible new structural design patterns in mind for both implementations we found that the decorator pattern implementation in TAC was more relevant and impactful to the overall design than any structural design pattern that we could conceive for Patrick and Lucas' game.\bigskip

\noin \underline{Design Pattern Justification and Incorporation:}\\
\tab We chose to incorporate the decorator design pattern into TAC. We observed that when a player buys fishing rods and ranger gear, similar behaviour occurs for each item added. The fishing skill or ranger skill is incremented and the best ranger gear or best fishing rod is set. In each case we can think of these additions and modifications as being `decorations' for the player. As such we can eliminate the need for the fishingSkill and rangerSkill int variables, and the bestRod and bestGear variables in the player. These can then be replaced with references to the classes that are being decorated - the BaseFishingSkill and BaseRangerSkill classes. \\
\tab Incorporating this design pattern required very little modification of the existing code. The Player class had to be adjusted to remove the old fishingSkill, rangerSkill, bestRod, and bestGear variables. The functionalities associated with the old variables in the Player class then had to be modified in order to interact with the new FishingSkill and RangerSkill classes. The only other place in the existing code that then had to be modified was the set of lines in the Controller class that incremented the fishingSkill and rangerSkill variables when the Player purchases a new rod or new gear.
\tab The addition of the Decorator pattern also took advantage of the existing implementation. It made use of the FishingRod and RangerGear enums in order to get all of the information necessary for the FishingSkill and RangerSkill updates. This meant that all different rods and gear with which the skills can be decorated can be given by the `AddRod' and `AddGear' classes. Due to the similarities in behaviour of each of rods and gear, this solution seemed much cleaner than creating separate concrete components for each type of rod and each type of ranger gear.\\
\tab Overall the process of incorporating this structural design pattern required little modification to the existing code. Going forward, it also provides an easy means to add new FishingRods, RangerGear, or other items that affect the Fishing and Ranger Skills. In these respect it adheres to the open-closed principle.


\end{document}